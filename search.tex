\documentclass{beamer}

\usepackage{beamerthemesplit} % // Activate for custom appearance

\usepackage{xeCJK}
\usepackage{graphicx}
\usepackage{subcaption}
\usepackage{tikz}
\usepackage{pgfplots}
\usepackage{algpseudocode}
\usepackage{listings}

\lstset
{
  breaklines=true,
  breakatwhitespace=true,
}

\definecolor{nthu}{HTML}{7F1084}
\definecolor{secondary}{HTML}{910A17}
\definecolor{accent}{HTML}{410A91}
\definecolor{bg}{HTML}{171717}

\setbeamercolor{normal text}{fg=white, bg=bg}
\setbeamercolor{alerted text}{fg=secondary}
\setbeamercolor{example text}{fg=accent}
\setbeamercolor{background}{parent=normal text}
\setbeamercolor{background canvas}{parent=background}
\setbeamercolor{palette primary}{fg=white, bg=secondary}
\setbeamercolor{palette secondary}{use=structure, fg=white, bg=nthu}
\setbeamercolor{palette tertiary}{use=structure, fg=white, bg=accent}
\setbeamercolor{block title}{fg=white, bg=nthu!69!accent}
\setbeamercolor{block body}{fg=white, bg=accent!69!nthu}
\setbeamercolor{titlelike}{fg=white, bg=nthu}
\setbeamercolor{structure}{fg=nthu!64}
\usebeamercolor{structure}
%\usecolortheme[named=nthu]{structure}
\useinnertheme{rounded}
\useoutertheme{infolines}
\usefonttheme{serif}

\xeCJKsetup{CJKglue=\hspace{0pt plus .12 \baselineskip }}
\xeCJKsetup{RubberPunctSkip=false}
\xeCJKsetup{PunctStyle=plain}
\xeCJKsetup{CheckSingle=true}
\XeTeXlinebreaklocale "zh"
\XeTeXlinebreakskip = 0pt plus 2pt

\setCJKmainfont{Songti TC}
\setCJKsansfont{Apple LiGothic}
\setmonofont{Cascadia Code PL}

\title{搜尋演算法與資料結構}
\subtitle{二分搜尋、二元搜尋樹⋯}
\author{nevikw39}
\institute{點石學園}
\date{\today}

\AtBeginSection{
	\frame
	{
%		\frametitle{Outline}
		\sectionpage
		\tableofcontents[sectionstyle=show/shaded, subsectionstyle=show/show/shaded, subsubsectionstyle=show/show/show/hide]
	}
}

\begin{document}

\frame{\titlepage}

\frame{\tableofcontents}

\section{Abstract}

\frame
{
	\frametitle{Introduction to Searching}
	
	\begin{definition}[Search]
		Find elements in an array such that some condition be satisfied.
	\end{definition}
	
	\pause
	
	\begin{itemize}
		\item If there is an element less than, equal to or greater than $x$??\\
		How many elements less than, equal to or greater than $x$??\pause
		\item The problem we are concerned about is the relation rather than order.
	\end{itemize}
}

\subsection{Linear Search}

\frame
{
	\frametitle{Pseudocode}
	
	\begin{algorithmic}
		\Procedure{Linear Search}{$*begin, *end, target$}
			\State$result\gets\infty$
            \For{$itr\in[begin, end)$}
            	\If{$*itr\geq target\land *itr < result$}
					\State$result\gets *itr$
				\EndIf
            \EndFor
            \State\Return$result$
        \EndProcedure
	\end{algorithmic}
}

\frame
{
	\frametitle{Efficiency}
	
	If there are plenty of queries, then \textsc{Linear Search} may not fit in time limit.\pause
	
	Therefore, we could do some preprocess such as sorting and utilize the monotonicity to improve the efficiency.
}

\section{Binary Search}

\frame
{
	\frametitle{Efficient Way to Search in Ordered Sequence}
	\framesubtitle{Divide and Conquer}
	
	\begin{description}
		\item<1->[Divide]Split array into two subarrays ($b = 2, O(1)$)
		\item<2->[Conquer]Search the subarray in which the target is ($a = 1$)
		\item<3->[Combine]Nothing to do
	\end{description}
}

\frame
{
	\begin{quotation}
		``Although the basic idea of binary search is comparatively straightforward, the details can be surprisingly tricky.''
	\end{quotation}
	\begin{flushright}
		--- Knuth, D. E.
	\end{flushright}
}

\subsection{Implementation}

\frame
{
	\frametitle{Pseudocode: Recursion}
	
	\begin{algorithmic}
		\Procedure{Binary Search}{$*begin, *end, target$}
		\If{$begin = end$}
			\State\Return$begin$
		\EndIf
		\State$mid\gets\frac{begin+end}{2}$
		\If{$target < *mid$}
			\State\Return\Call{Binary Search}{begin, mid, target}
		\ElsIf{$target = *mid$}
			\State\Return$mid$
		\Else
			\State\Return\Call{Binary Search}{mid+1, end, target}
		\EndIf
        \EndProcedure
	\end{algorithmic}
}

\frame
{
	\frametitle{Tail Call \& Tail Recursion}
	
	\begin{definition}
		It's called \textbf{Tail Call} that the return statement of a function is to call some function.
		
		If the called one is the calling one itself, it's called \textbf{Tail Recursion}.
	\end{definition}
	
	\pause
	
	On top of the fact that recursion requires quite a few resources, we could use loop to attain better performance.
}

\frame
{
	\frametitle{Pseudocode: Loop}
	
	\begin{algorithmic}
		\Procedure{Binary Search}{$*begin, *end, target$}
		\While{$begin\neq end$}
			\State$mid\gets\frac{begin+end}{2}$
			\If{$target < *mid$}
    			\State$end\gets mid$
    		\ElsIf{$target = *mid$}
    			\State\Return{mid}
    		\Else
    			\State$begin\gets mid+1$
			\EndIf
		\EndWhile
		\State\Return$begin$
        \EndProcedure
	\end{algorithmic}
}

\frame{
	\frametitle{Example}
	\framesubtitle{$target=110$}
	
	\begin{center}
        \begin{tikzpicture}[scale=0.85, transform shape]
        	\only<1>{
        		\foreach \x/\val/\col in {0/8/accent,1/20/gray,2/20/gray,3/27/gray,4/27/nthu,5/27/gray,6/110/gray,7/110/gray,8/2021/gray,9//secondary}{
        			\node[draw=\col,rectangle, fill=\col!92, minimum size =1cm] (c) at (\x,0) {\val};
        		}
        	}
			\only<2>{
        		\foreach \x/\val/\col in {0/8/gray,1/20/gray,2/20/gray,3/27/gray,4/27/accent,5/27/gray,6/110/nthu,7/110/gray,8/2021/gray,9//secondary}{
        			\node[draw=\col,rectangle, fill=\col!92, minimum size =1cm] (c) at (\x,0) {\val};
        		}
        	}
			\only<3>{
        		\foreach \x/\val/\col in {0/8/gray,1/20/gray,2/20/gray,3/27/gray,4/27/gray,5/27/gray,6/110/nthu,7/110/gray,8/2021/gray,9//gray}{
        			\node[draw=\col,rectangle, fill=\col!92, minimum size =1cm] (c) at (\x,0) {\val};
        		}
        	}
		\end{tikzpicture}
	\end{center}
}

\subsubsection{Alternative Approach}

\frame
{
	\frametitle{Pseudocode}
	
	\begin{algorithmic}
		\Procedure{Binary Search}{$*begin, *end, target$}
    		\For{$n\gets end-begin, jump\gets n/2; jump>0; jump\gets jump/2$}
    			\While{$begin+jump<end\land *(begin+jump)<target$}
    				\State$begin\gets begin+jump$
    			\EndWhile
    		\EndFor
			\State\Return$begin+1$
        \EndProcedure
	\end{algorithmic}
}

\frame{
	\frametitle{Example}
	\framesubtitle{$target=110$}
	
	\begin{center}
        \begin{tikzpicture}[scale=0.85, transform shape]
        	\only<1>{
        		\foreach \x/\val/\col in {0/8/accent,1/20/gray,2/20/gray,3/27/gray,4/27/secondary,5/27/gray,6/110/gray,7/110/gray,8/2021/gray}{
        			\node[draw=\col,rectangle, fill=\col!92, minimum size =1cm] (c) at (\x,0) {\val};
        		}
        	}
			\only<2>{
        		\foreach \x/\val/\col in {0/8/gray,1/20/gray,2/20/gray,3/27/gray,4/27/accent,5/27/gray,6/110/gray,7/110/gray,8/2021/secondary}{
        			\node[draw=\col,rectangle, fill=\col!92, minimum size =1cm] (c) at (\x,0) {\val};
        		}
        	}
			\only<3>{
        		\foreach \x/\val/\col in {0/8/gray,1/20/gray,2/20/gray,3/27/gray,4/27/accent,5/27/gray,6/110/secondary,7/110/gray,8/2021/gray}{
        			\node[draw=\col,rectangle, fill=\col!92, minimum size =1cm] (c) at (\x,0) {\val};
        		}
        	}
			\only<4>{
        		\foreach \x/\val/\col in {0/8/gray,1/20/gray,2/20/gray,3/27/gray,4/27/accent,5/27/secondary,6/110/gray,7/110/gray,8/2021/gray}{
        			\node[draw=\col,rectangle, fill=\col!92, minimum size =1cm] (c) at (\x,0) {\val};
        		}
        	}
			\only<5>{
        		\foreach \x/\val/\col in {0/8/gray,1/20/gray,2/20/gray,3/27/gray,4/27/gray,5/27/nthu,6/110/gray,7/110/gray,8/2021/gray}{
        			\node[draw=\col,rectangle, fill=\col!92, minimum size =1cm] (c) at (\x,0) {\val};
        		}
        	}
		\end{tikzpicture}
	\end{center}
}

\subsection{Built-in Functions}

\frame
{
	\frametitle{\ttfamily std::lower\_bound()}
	\framesubtitle{Find the minimum element that is no less than (greater than or equal to) $target$}
	
	\lstinputlisting[language=C++, firstline=36, lastline=40]{04_binary_search_builtin.cpp}
}

\frame
{
	\frametitle{\ttfamily std::upper\_bound()}
	\framesubtitle{Find the minimum element that is greater than $target$}
	
	\lstinputlisting[language=C++, firstline=41, lastline=45]{04_binary_search_builtin.cpp}
}

\frame{
	\frametitle{Examples}
	\framesubtitle{$target=110$}
	
	\begin{center}
        \begin{tikzpicture}[scale=0.85, transform shape]
        	\only<1>{
        		\foreach \x/\val/\col in {0/8/gray,1/20/gray,2/20/gray,3/27/gray,4/27/gray,5/27/gray,6/110/gray,7/110/gray,8/2021/gray}{
        			\node[draw=\col,rectangle, fill=\col!92, minimum size =1cm] (c) at (\x,0) {\val};
        		}
        	}
        	\only<2>{
        		\foreach \x/\val/\col in {0/8/accent,1/20/accent,2/20/accent,3/27/accent,4/27/accent,5/27/accent,6/110/gray,7/110/gray,8/2021/gray}{
        			\node[draw=\col,rectangle, fill=\col!92, minimum size =1cm] (c) at (\x,0) {\val};
        		}
        	}
        	\only<3>{
        		\foreach \x/\val/\col in {0/8/accent,1/20/accent,2/20/accent,3/27/accent,4/27/accent,5/27/accent,6/110/nthu,7/110/nthu,8/2021/gray}{
        			\node[draw=\col,rectangle, fill=\col!92, minimum size =1cm] (c) at (\x,0) {\val};
        		}
        	}
        	\only<4>{
        		\foreach \x/\val/\col in {0/8/accent,1/20/accent,2/20/accent,3/27/accent,4/27/accent,5/27/accent,6/110/nthu,7/110/nthu,8/2021/secondary}{
        			\node[draw=\col,rectangle, fill=\col!92, minimum size =1cm] (c) at (\x,0) {\val};
        		}
        	}
		\end{tikzpicture}
	\end{center}
}

\frame
{
	\frametitle{Examples}
	\framesubtitle{\ttfamily std::lower\_bound()}
	
	If vector $v$ is sorted:\pause
	
	\begin{itemize}
		\item Range of elements that less than $x$, $\{i\ |\ \forall i\in v\implies i<x\}$:\\\pause
		$[v.begin(), lower\_bound(v.begin(), v.end(), x))$\pause
		\item Range of elements that no less than (greater than or equal to) $x$, $\{i\ |\ \forall i\in v\implies i\nless(\geq) x\}$:\\\pause
		$[lower\_bound(v.begin(), v.end(), x), v.end())$
	\end{itemize}
}

\frame
{
	\frametitle{Examples}
	\framesubtitle{\ttfamily std::upper\_bound()}
	
	If vector $v$ is sorted:\pause
	
	\begin{itemize}
		\item Range of elements that no greater than (less than or equal to) $x$, $\{i\ |\ \forall i\in v\implies i\ngtr(\leq) x\}$:\\\pause
		$[v.begin(), upper\_bound(v.begin(), v.end(), x))$\pause
		\item Range of elements that greater than $x$, $\{i\ |\ \forall i\in v\implies i>x\}$:\\\pause
		$[upper\_bound(v.begin(), v.end(), x), v.end())$	\end{itemize}
}

\frame
{
	\frametitle{Examples}
	
	If vector $v$ is sorted:\pause
	
	\begin{itemize}
		\item Range of elements that equal to $x$, $\{i\ |\ \forall i\in v\implies i = x\}:$\\\pause
		$[lower\_bound(v.begin(), v.end(), x), upper\_bound(v.begin(), v.end(), x))$\pause
		\item Range of elements that greater than $x$ and less than $y$, $\{i\ |\ \forall i\in v\implies x < i < y\}:$\\\pause
		$[upper\_bound(v.begin(), v.end(), x), lower\_bound(v.begin(), v.end(), y))$\pause
		\item Range of elements that greater than or equal to $x$ and less than or equal to $y$, $\{i\ |\ \forall i\in v\implies x\leq i\leq y\}:$\\\pause
		$[lower\_bound(v.begin(), v.end(), x), upper\_bound(v.begin(), v.end(), y))$
	\end{itemize}
}

\frame
{
	\frametitle{Small Factor}
	\framesubtitle{UVa 11621}
	
	We could construct $C_{2,3}$ first. For every $m$ inputted, the answer is $*lower\_bound(c.begin(), c.end(), m)$.
}

\frame
{
	\frametitle{Exact Sum}
	\framesubtitle{UVa 11057}
	
	How could we determine whether there exist any exact value $v$ in an ordered sequence $s$??\pause
	
	\begin{itemize}
		\item The range of elements that equal to $v$ is $[lower\_bound(s.begin(), s.end(), v), upper\_bound(s.begin(), s.end(), v))$.\pause
		\item If the length of range (i.e. $upper\_bound(s.begin(), s.end(), v) - lower\_bound(s.begin(), s.end(), v)$) is greater than $0$, then it means $v$ occurs at least one time.
	\end{itemize}
}

\frame
{
	\frametitle{Triangles}
	\framesubtitle{AtCoder Beginner Contest 143 p. D}
	
	\begin{block}{Condition to Construct a Triangle}
		Let $a, b, c$ be the three edges of the triangle, where $a\leq b\leq c$, then $b - a < c < a + b$
	\end{block}
	
	\pause
	
	Lest we may count duplicate ones, we could enumerate the shorter edges and search the range of the third edge, whose closed left end is upper bound of $b-a$ and opened right end is lower bound of $a+b$.\pause
	
	Note that because $c$ should be the longest edge, we mustn't search from $v.begin()$ but from the next iterator point to $b$.
}

\frame
{
	\frametitle{Tsundoku}
	\framesubtitle{AtCoder Beginner Contest 172 p. C}
	
	\begin{block}{Problem}
		Find the max $i+j$ such that $\sum_{k=0}^iA_k + \sum_{l=0}^jB_l \leq K$.
	\end{block}
	
	\pause
	
	\begin{block}{Prefix Sum}
		We could define $PSA_i\gets\sum_{k=0}^iA_k$, $PSB_j\gets\sum_{l=0}^jB_l$.\pause
		
		If $\forall i\in A, i\geq 0\land\forall i\in B, i\geq 0$, then it's obviously that the prefix sum of $A$ and $B$ is increasing.\pause
	\end{block}
	
	Thus, we could calculate the prefix sum of $A$ and $B$ first, and for all $i$ in the prefix sum of $A$, we perform \textsc{Binary Search} to find the max $j$ in the prefix sum of $B$ such that $i+j\leq K$ i.e. $j\leq K-i$.
}

\subsubsection{Useful Technique: Coordinate Compression}

\frame
{
	\frametitle{Coordinate Compression}
	
	If the value range of input data is quite enormous and we care about not the exact values but their relative relation, we could apply a technique called \textbf{Coordinate Compression}, mapping the value in codomain into $[0, n)$.
}

\frame
{
	\frametitle{Pseudocode}
	
	\begin{algorithmic}
		\small
		\Procedure{Coordinate Compression}{$v$}
    		\State$rk\gets v, cc\gets \{\}$
			\State\Call{Sort}{rk.begin(), rk.end()}
			\State\Call{Unique}{rk.begin(), rk.end()}\Comment{Remove duplicate elements in $rk$}
			\For{$i\in v$}
				\State$cc.push\_back($\Call{Lower Bound}{$rk.begin(), rk.end(), i$}$ - rk.begin())$
			\EndFor
			\State\Return$cc$
        \EndProcedure
	\end{algorithmic}
}

\frame{
	\frametitle{Examples}
	
	\begin{center}
        \begin{tikzpicture}[scale=0.85, transform shape]
        	\only<1->{
        		\foreach \x/\val/\col in {0/8/accent,1/27/nthu,2/2021/secondary,3/110/nthu!50!secondary,4/20/accent!50!nthu,5/110/nthu!50!secondary,6/27/nthu,7/20/accent!50!nthu,8/27/nthu}{
        			\node[draw=\col,rectangle, fill=\col!92, minimum size =1cm] (c) at (\x,0) {\val};
        		}
        	}
        	\only<2>{
        		\foreach \x/\val/\col in {0/0/accent,1/2/nthu,2/4/secondary,3/3/nthu!50!secondary,4/1/accent!50!nthu,5/3/nthu!50!secondary,6/2/nthu/,7/1/accent!50!nthu,8/2/nthu}{
        			\node[draw=\col,rectangle, fill=\col!92, minimum size =1cm] (c) at (\x,-2) {\val};
        		}
        	}
		\end{tikzpicture}
	\end{center}
}

\frame
{
	\frametitle{\ttfamily std::unique()}
	\framesubtitle{Move duplicate elements in an ordered sequence to end and return new valid end}
	
	\lstinputlisting[language=C++, firstline=15, lastline=17]{10_coordinate_compression.cpp}
}

\frame
{
	\frametitle{Reorder Cards}
	\framesubtitle{AtCoder Beginner Contest 213 p. C}
	
	\begin{quote}
		``It can be proved that these positions are uniquely determined without depending on the order in which the operations are done.''
	\end{quote}
	
	\pause
	
	Hence, rows and columns could be reordered separately.\pause
	
	It's obvious that any operation cannot affect the relative position of cards.\pause
	
	At the end of process, the coordinate of cards must be in $[1, n]$.\pause
	
	Voil\`{a}! All we have to do is to perform coordinate compression in both $x$ and $y$ axis.
}

\subsection{Binary Search in Function}

\frame
{
	It's not hard for us to refer to array as a sort of function.\pause
	
	Then if we could perform \textsc{Binary Search} in monotonic array, couldn't we also perform \textsc{Binary Search} in monotonic function??
}

\frame
{
	\frametitle{Buy an Integer}
	\framesubtitle{AtCoder Beginner Contest 146 p. C}
	
	Apparently, $\forall i\geq 0, d(i)\leq d(i+1)$.\pause
	
	Therefore, it could be easily proven that $\forall i\geq 0, price(i) = A\times i+B\times d(i) < A\times(i+1)+B\times d(i+1) = price(i+1)$.\pause
}

\frame
{
	\frametitle{Logs}
	\framesubtitle{AtCoder Beginner Contest 174 p. E}
	
	The minimum of shortest possible length $x$ is $1$, and the maximum is the maximum length of logs.\pause
	
	Moreover, whether we could cut all logs to less than $x$ in at most $k$ times is monotonic.
}

\section{Binary Search Tree}

\frame
{
	\frametitle{Dynamic Problems}
	
	Some problems require us not only to answer some queries but also to modify the original data alternately.
	
	If we sort the data each time we do modification, the time complexity would decline to $O(Q\times N\log N)$.
	
	As a result, we need some data structure to avoid that.
}

\frame
{
	\frametitle{Self-Balancing Binary Search Tree}
	
	\begin{description}
		\item[Tree] is a kind of abstract data type on which we could easily insert or traverse.\pause
		\item[Binary Tree] is a kind of rooted tree that each node could have at most two child nodes.\pause
		\item[Binary Search Tree] is a kind of binary tree where the value of internal node always greater than all values in its left subtree and less than all values in its right subtree.\pause
		\item[Self-Balancing Binary Search Tree] is a BST which could balance its height by change root and/or rotating so that the max hegiht won't exceed $O(\log N)$.
	\end{description}
}

\subsection{std::set<>}

\frame
{
	\frametitle{Set in Math \& C++}
	
	In math:
	\begin{itemize}
		\item<1-> The ordering of a set doesn't matter; that is, $\{8, 27\} = \{27, 8\}$.
		\item<2-> Each element in set should be unique.
	\end{itemize}
	
	In C++:
	\begin{itemize}
		\item<1-> Because the order of set is relevant, we could choose a unique order for representation. In the set of C++, the elements are sorted by their relation.
		\item<2-> Each element in set should be unique.
	\end{itemize}
}

\frame
{
	\frametitle{Usage}
	
	Like \texttt{std::vector<>}, \texttt{std::set<>} is a kind of template class container in C++ Standard Template Library.
	
	The most difference is that we couldn't use \texttt{[]} operator to get the access of elements.
	
	In addition, We could still use iterators to traverse all elements, but the iterators are bidirectional, which means they doesn't support \texttt{+ n}, \texttt{- n} operations.
}

\frame
{
	\frametitle{Usage (cont.)}
	
	\begin{description}
		\item[\ttfamily s.insert()] Add a new element.
		\item[\ttfamily s.erase()] Erase an element by its value or iterator.
		\item[\ttfamily s.find()] If the element exists in the set then return its iterator, else return \texttt{s.end()}.
		\item[\ttfamily s.lower\_bound()] The specific version of \texttt{std::lower\_bound()}.
		\item[\ttfamily s.upper\_bound()] The specific version of \texttt{std::upper\_bound()}.
	\end{description}
}

\frame
{
	\frametitle{Problems to Practice}
	
	\begin{itemize}
		\item AtCoder Beginner Contest 164 p. C - gacha
		\item AtCoder Beginner Contest 073 p. C - Write and Erase
	\end{itemize}
}

\frame
{
	\frametitle{Cutting Woods}
	\framesubtitle{AtCoder Beginner Contest 217 p. D}
	
	How could we know the length of the segment a mark is in??\pause
	
	\begin{itemize}
		\item We could find the greatest cut mark on its left and the least one on its right.\pause
		\item Be sure to insert $0$ and $L$ first so that we could avoid boundary  conditions.
	\end{itemize}
}

\subsection{std::map<,>}

\frame
{
	\frametitle{Map, dictionary or associative array}
	
	Sometimes, we need to store extra values related to the nodes in a binary search tree.
	
	\texttt{std::map<,>} is a C++ STL container that allow us to store data in that way.
	
	The data used to build binary search tree are called \textbf{keys}, and the extra data related to some keys are called \textbf{values}.
	
	In fact, a \texttt{std::set<>} could be deemed \texttt{std::map<,>} without \textbf{values} the extra data.
}

\frame
{
	\frametitle{Usage}
	
	Just like \texttt{std::set<>}, the type of \textbf{key} or \textbf{value} could be any type if it supports \texttt{<, ==} operations.
	
	We could use \texttt{[]} operator to access the \textbf{value} of a \textbf{key}.
	
	Iterators also work with \texttt{std::map<,>}, and the key is \texttt{itr->first} while the value is \texttt{itr->second}.
}

\frame
{
	\frametitle{Usage (cont.)}
	
	\begin{description}
		\item[\ttfamily m[{]}] Return the reference to the \textbf{value} of a \textbf{key}.
		\item[\ttfamily m.erase()] Erase an element by its \textbf{key} or iterator.
		\item[\ttfamily m.find()] If the key exists in the map then return its iterator, else return \texttt{s.end()}.
		\item[\ttfamily m.lower\_bound()] The specific version of \texttt{std::lower\_bound()}.
		\item[\ttfamily m.upper\_bound()] The specific version of \texttt{std::upper\_bound()}.
	\end{description}
}

\frame
{
	\frametitle{Problems to Practice}
	
	\begin{itemize}
		\item UVa 10420 - List of Conquests
		\item UVa 10226 - Hardwood species
		\item UVa 12250 - Language Detection
		\item AtCoder Beginner Contest 155 p. C - Poll
		\item AtCoder Beginner Contest 137 p. C - Green Bin
	\end{itemize}
}

\subsection{std::multiset<>, std::multimap<,>}

\frame
{
	\frametitle{\ttfamily std::multiset<>, std::multimap<,>}
	
	The containers that accept duplicate keys.
	
	Note that we cannot use \texttt{[]} operator in \texttt{std::multimap<,>}.
}

\section{Hash Functions \& Tables}

\frame
{
	\frametitle{Hash Functions}
	
	A \textsc{Hash Function} is a function that map some data into value in fixed \textit{range}.
	
	The \textit{range} of inputted data may be large, even infinite, whereas the \textit{range} of \textsc{Hash Function} is fixed.
	
	As a consequence, when it come to searching, instead of searching the whole \textit{range} of inputted data, we could search in the \textit{range} of \textsc{Hash Function}.
}

\frame
{
	\frametitle{Hash Functions (cont)}
	
	Note that on top of the fact that \textsc{Hash Function} may not hold monotonicity, we could only search whether the \textit{key} exist.
	
	In addition, since the \textit{range} of \textsc{Hash Function} is finite, the \textit{hash values} of different \textit{key} may be the same, which is called \textsc{Collision}.
}

\frame
{
	\frametitle{Hash Container}
	
	C++ STL provides \texttt{unordered\_set} and \texttt{unordered\_map} which is the hash version of \texttt{set} and \texttt{map} respectively.
	
	The usage is mostly the same, the main difference is that unordered container cannot perform \textsc{Binary Search}.
}

\end{document}
